\documentclass{journalofficiel}

\addto\captionsfrench{
  \title{Journal officiel n°0007}
  \author{Démocratie Normalienne Ernestienne et Supercool}
}

\addto\captionsernestien{
  \title{lelalezedefsê librêg kâzg}
  \author{nuranjêrs ernestôg bizârg feltanîg}
}

\date{\today}

\begin{document}
\passageLangue{french}

\part{Révisions constitutionnelles}
\section{Dénomination des citoyens permanents}
Sont renommés les status suivants:
\begin{itemize}
  \item \textit{citoyen permanent} en \textit{conseiller permanent}
  \item \textit{citoyen honoraire} en \textit{conseiller honoraire}
  \item \textit{citoyen honoraire²} en \textit{citoyen honoraire émérite}
\end{itemize}

\section{Statut municipal}
Le statut de maire est créé. Ses prérogatives sont inscrites dans le code communal. C'est un responsable administratif, mais qui n'est pas nommé par le conseil constitutionnel.

\section{Résidence}
L'article 3.3.1 est supprimé, est ainsi la notion de résidence n'est plus présente dans la constitution. Les anciens citoyens sont renommés citoyens actifs.

\section{Report des élections de janvier}
Les prochaines élections du conseil constitutionnel sont reportées de deux semaines en raison de la coupe du monde.

\section{Revendications territoriales temporaires}
L'Ernestie revendique, en plus du bassin aux Ernests, les flaques présentes en Courô les jours de pluie.

\newpage
\part{Nouvelles lois}
\section{Code général communal}
Le code général communal est adopté. Son contenu est présent sur le serveur.

\section{Prix spécial du sauvetage de gnocchis}
Ce prix nouvellement créé est remis aux ernestiens s'illustrants dans ce domaine. Il est aujourd'hui remis à @petit\_yaourt.

\section{Double-comptes}
Une personne physique ne peut être avoir plusieurs identités virtuelles citoyennes simultanément.

\section{Humour normalien dans les commerces}
Les commerces doivent avoir un jeu de mots ernestien dans leur nom. Les contrevenants devront, à partir du texte du poisson Steve en esperanto et en dibi, le traduire en ernestien et l'interpréter.

\part{Nombre de membres}
L'Ernestie a dorénavant plus de 1 000 000 membres (en base 2).

\passageLangue{ernestien}
\part{plupareîfi biblêg}
\section{alâz dâ librêgzi}
plupareîq alâz dâ zîs jebokûi:
\begin{itemize}
  \item \textit{librêgz} frêd \textit{biblêgz malpapiiôg}
  \item \textit{vioglibrêgz} frêd \textit{biblêgz malpapiiôg viôg}
  \item \textit{vioglibrêgz ranjjâz} frêd \textit{biblêgz malpapiiôg viôg viôg}
\end{itemize}

\section{diudêf dâ jebokû dâ ehârp}
diûdq jebokû dâ ehârp. zedpôp parîsg larûsq skilfêi sîtg. sît êq enârk, sêd biblêgzi rokonseîq sît.

\section{nîr}
alapubêl lelale 3.3.1.

\section{dûblef dâ plufplûf dâ biblêgzi}
dûblq plufplûf dâ biblêgzi frêd pôst 11 nuîi.

\section{seamuâf erikêg papiiôg}
ernestô seamuâq flikflôki în kurô.

\newpage
\part{pôpi konskrîg}
\section{zedpôp parîsg}
rienâq în diskôrd

\section{jtelôfr dâ pompiêf dâ aidezûrsi}
ernestôgzi, bît selâ pompiêq aidezûrsi, bienvenûq zîs jtelôfr. \foreignlanguage{french}{@petit\_yaourt} bienvenûq zîs jtelôfr.

\section{ementâlgr uljâz}
bâz populâs rokâpq â segpâ ementâlgr bît selâ êq unîgz.

\section{fenlapuâr bizârg în ehôp}
ehôp ernestôg aavekûq risteiêr ernestinbûhg în alâz sîtg.

\part{palataî dâ populâsi}
êq 1000000 \foreignlanguage{french}{(}în vunavepâ jazg\foreignlanguage{french}{)} populâsi în ernestô.

\end{document}
