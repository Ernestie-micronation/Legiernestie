\documentclass{journalofficiel}

\addto\captionsfrench{
  \title{Journal officiel n°0006}
  \author{Démocratie Normalienne Ernestienne et Supercool}
}

\date{\today}

\begin{document}
\passageLangue{french}

\part{Révisions constitutionnelles}
\section{Ajout d'un titre à la constitution}
L'article 5 de la constitution est déplacé dans sa propre partie, nomméé "Conseil constitutionnel".

\section{Précisions concernant le renouvellement partiel du conseil constitutionnel}
S'ajoute à l'article 5.4.3.b, la phrase suivante:
\emph{Pendant une élection de renouvellement partiel, les votes du premier et second tour pourront être initiés d'autres jours que le mercredi ou dimanche ; seules les durées relatives des votes et de l'entre deux tour, de 48h doivent être respectées. Ces votes devront toujours être initiés en journée.}

\section{Modification des règles de censure}
Le terme volontairement est supprimé de l'article 16.2. Ainsi, le conseil constitutionnel peut dorénavant censurer un texte prêtant involontairement à confusion.

\section{Ajout du vote blanc}
Le vote blanc est désormais inscrit dans la constitution. Les personnes votant blanc seront reconnus comme participant à la vie de l'Ernestie, notamment pour obtenir le statut d'ancien citoyen. Le vote blanc n'influence pas le résultat du vote.

\section{Redéfinition du statut d'ancien citoyen}
Les modalités nécessaires pour devenir ancien citoyen sont réduites à avoir participé à au moins 3 des 10 derniers votes clos.
\newpage
\part{Nouvelles lois}

\section{Ecriture d'un traité de reconnaissance mutuelle avec la République Participative du Dibistan}
Les citoyens ernestiens sont favorables à l'ecriture d'un Traité de Reconnaissance Mutuelle avec la République Participative du Dibistan. Celui-ci sera rédigé une fois que les citoyens de la RPD auront donné leur accord.

\section{Plat officiel de l'Ernestie}
La présente loi fait du plateau de fromage le plat officiel de l'Ernestie. Ce plateau de fromage devra contenir au moins la moitié des fromages suivants : 
\begin{itemize}
  \item Saint Marcellin
  \item Saint-Nectaire
  \item Roquefort
  \item Mont d'or
  \item Brillat Savarin
  \item Mimolette
  \item Tomme de Savoie
  \item Camembert
  \item Époisse
  \item Comté
  \item Picodon
\end{itemize}
D'autres fromages peuvent être ajoutés. Tous les fromages peuvent être remplacés par leur version vegan. Lors des évènements officiels, il est nécessaire d'avoir au moins un fromage vegan.

\newpage
\part{Lois refusées}
\section{Ecriture des textes officiels en jaune-violet}
Cette proposition de loi a été refusée avec 3 votes pour et 9 votes contre.

\end{document}
