\documentclass[fontsize=14pt, DIV=calc]{scrartcl}

\usepackage[english, french]{babel}
\usepackage{graphicx}


\babelprovide{esperanto}
\babelprovide[import=fr, hyphenrules = english]{ernestien}
\babelfont[ernestien]{rm}{ErnestFont}
\babelfont[ernestien]{sf}{ErnestFont}
\babelfont[ernestien]{tt}{ErnestFont}

\newcommand{\ern}[1]{\foreignlanguage{ernestien}{#1}}
\newcommand{\en}[1]{\foreignlanguage{english}{#1}}
\newcommand{\es}[1]{\foreignlanguage{esperanto}{#1}}



\begin{document}
\begin{tabular}{c c}
    \includegraphics[scale=0.8]{esperlando.png} & \includegraphics[scale=0.35]{drapeau.png}
\end{tabular}
\begin{center}
\bigskip
\huge{
    \es{Traktato de reciproka agnosko inter:}

    \ern{pabiologîk dâ uisevrêf jakfredjâg:}

    Traité de reconnaissance mutuelle entre:

    \en{Treaty of mutual recognition between:}

}
\bigskip
\bigskip
\bigskip
\bigskip

\Large{
\es{La Virtuala Princlando de Esperlando 

kaj} \ern{aê}

\ern{nuranjêrs ernestôg bizârg feltanîg}
}

\end{center}
\newpage
\es{Per ĉi tiu traktato, ni, subskribaj}

\ern{kôn zîs pabiolojîk, ludî }

Par ce traité, nous, soussignés

\en{By this treaty, we, undersigned}

\begin{center}
    \es{Princino Alenia li Ia Reïozy-De Esperlando

    kaj} \ern{aê}

    \ern{dihellgô, lezamôgz ernestôg în esperlandô}

\end{center}

\es{deklaras ke la nacioj ke ni reprezentas reciproke agnoskas. Ni kreos ambasadejoj por ni reciproke en nia respektiva lando.}

\ern{epsilonuzâng âkq zî uniî bît ludî uisenûq uisevrêq jakfredjâg. ludî diûdqq jakfredjâg lezamoî în ponî ludîg.}

déclarons que les nations que nous représentons se reconnaissent mutuellement. Nous créerons des ambassades l'un pour l'autre dans nos pays respectifs.

\en{declare that the nations that we represent recognize each other. We will create embassies for each other in our respective countries.}

\end{document}
